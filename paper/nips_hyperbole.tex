% 
% Annual Cognitive Science Conference
% Sample LaTeX Paper -- Proceedings Format
% 

% Original : Ashwin Ram (ashwin@cc.gatech.edu)       04/01/1994
% Modified : Johanna Moore (jmoore@cs.pitt.edu)      03/17/1995
% Modified : David Noelle (noelle@ucsd.edu)          03/15/1996
% Modified : Pat Langley (langley@cs.stanford.edu)   01/26/1997
% Latex2e corrections by Ramin Charles Nakisa        01/28/1997 
% Modified : Tina Eliassi-Rad (eliassi@cs.wisc.edu)  01/31/1998
% Modified : Trisha Yannuzzi (trisha@ircs.upenn.edu) 12/28/1999 (in process)
% Modified : Mary Ellen Foster (M.E.Foster@ed.ac.uk) 12/11/2000
% Modified : Ken Forbus                              01/23/2004
% Modified : Eli M. Silk (esilk@pitt.edu)            05/24/2005
% Modified: Niels Taatgen (taatgen@cmu.edu) 10/24/2006

%% Change ``a4paper'' in the following line to ``letterpaper'' if you are
%% producing a letter-format document.

\documentclass{article} % For LaTeX2e

\usepackage{nips12submit_e,times}
\usepackage{pslatex}
\usepackage{amsmath}
\usepackage{amsfonts}
\usepackage{latexsym}
\usepackage{amssymb}
\usepackage{apacite}
\usepackage{graphicx}
\usepackage{xspace}
\usepackage{multirow}
\usepackage{array}
\usepackage{caption}
\usepackage{subcaption}

\newcommand{\dictionary}{\ensuremath{\mathcal{D}}\xspace}
\newcommand{\citey}{\cite}

\title{Halo, Hyperbole, and the Pragmatic \\ Interpretation of Numbers}
 
\author{
Jean Y. Wu \\
Symbolic Systems Program\\
Stanford University\\
Stanford, CA 94305 \\
\texttt{jeaneis@stanford.edu} \\
\And
Justine T. Kao \\
Department of Psychology\\
Stanford University \\
Stanford, CA 94305 \\
\texttt{justinek@stanford.edu} \\
\AND
Leon Bergen \\
Department of Brain and Cognitive Sciences\\
Massachusetts Institute of Technology \\
Cambridge, MA 02138\\
\texttt{bergen@mit.edu} \\
\And
Noah D. Goodman \\
Department of Psychology\\
Stanford University \\ 
Stanford, CA 94305\\
\texttt{ngoodman@stanford.edu} \\
}


\begin{document}

\maketitle

\begin{abstract}
[REWRITE] 

\textbf{Keywords:} 
Number interpretation, hyperbole, pragmatic halo, pragmatics, Bayesian model
\end{abstract}


\section{Introduction}

The meanings of numbers are usually thought to be fixed and precise. In mathematics, the number $30$ has a precise meaning that clearly distinguishes it from numbers like $32$ and $1000$. In everyday language, however, numbers are treated much more flexibly: people do not always mean what they say when using number words. In this paper we examine the pragmatic interpretation of number words using formal models and behavioral experiments.
%the kinds of inference mechanisms that people utilize in order to identify when a number is not intended to be interpreted literally. 
In particular, we explore the phenomena of \emph{pragmatic halo} and \emph{hyperbole.} Halo refers to imprecise use of number words (for instance, ``I'll be there in 30 minutes'' means \emph{about} 30 minutes), hyperbole refers to the use of exaggeration to convey affective subtext (for instance, ``I waited for a million hours'' means roughly ``I waited for a while and I didn't like it'').
Building on recent models of pragmatics as rational inference by a speaker and listener about each other, we propose a computational model that captures both halo and hyperbole effects, as well as their interaction.

Some numbers tend to be interpreted as more precise than others \cite{Lasersohn}. Suppose a friend tells you: ``I will be there in $30$ minutes." The number $30$ can be interpreted to mean somewhere within the range of $25$ and $40$, depending on how punctual your friend usually is. On the other hand, if your friend says: ``I will be there in $32$ minutes," it is more likely that you would interpret the number to mean exactly $32$. 
%The intuition behind this is that if your friend intended to convey the meaning of ``somewhere around $32$", she would have uttered $30$ instead, because $30$ conveys the same meaning and is less effortful to utter. 
[Krifka et al] described how this \emph{pragmatic halo} effect can be explained under the assumption that speakers prefer number expressions that are shorter and less costly to utter, listeners will then favor approximate interpretations of round numbers even if there is no general bias for approximate interpretations. [Bastiaanse] further argued that interpreting round numbers as approximate is a rational choice. Here we formalize these arguments within a Bayesian framework for pragmatic inference: if number words could be interpreted precisely or approximately, and some number words are more costly than others, we show that a rational listener will interpret the more costly number words as more precise.
%Here we refer to this effect as the \emph{pragmatic halo}, where certain numbers tend to be interpreted with less precision than others. For example, the possible interpretations of the number $30$ is a distribution centered around $30$.

Not only can numbers be used loosely, they can also be used hyperbolically. Suppose you overhear a student saying, ``It takes 30 minutes to scroll down that professor's list of publications!" Given that it is very unlikely for the literal meaning of the utterance to be true, the number $30$ in this utterance is not likely to be interpreted literally. Instead, the actual amount of time it takes to scroll down the professor's list of publications is more likely to be interpreted as much less than 30 minutes, though still greater than average. Moreover, an affective subtext is likely to be conveyed: the list of publications is strikingly long.
Hyperbolic utterances often express important interpersonal meaning beyond the literal meaning of the statement, and successful interpretation of such expressions hinges on the listener's ability to infer the speaker's intentions \cite{mccarthy2004there, gibbs2000irony, cano2003risk}. Previous work has focused on cues for verbal irony and exaggeration, such as a slow speaking rate, heavy stress, nasalization, and interjections \cite{kreuz1995two, kreuz2007lexical}. Although lexical and prosodic information has been shown to be important for both human and machine detection of hyperbole \cite{davidov2010semi, reyes2011mining, van2007algorithm}, we argue that common prior knowledge about the relevant topic
%the distribution of numerical values associated with the relevant topic 
also plays an important role in identification and interpretation of hyperbolic statements. That is, part of what makes a statement likely to receive a hyperbolic interpretation is that both speaker and listener know the literal meaning is very unlikely. 
While a publication list that takes thirty minutes to scroll is very unlikely, waiting for a friend for thirty minutes is not; correspondingly, ``30 minutes'' is likely to be interpreted as hyperbolic exaggeration for scrolling times, but as approximately literal for waiting times.
%For example, suppose that the cost of a tall latte at Starbucks is always $2.75$ dollars, and suppose someone who went to Starbucks every day told you, ``That Starbucks tall latte cost, like, two dollars and seventy-five cents!" Even if the statement was uttered with many interjections, a slow speaking rate, heavy stress, and nasalization, it would be difficult to interpret the utterance as being hyperbolic, because the distribution of numerical values associated with Starbucks tall lattes is a single value at $2.75$. As a result, we propose that the statistical properties and distributions of numbers in the natural world should serve as a major cue for hyperbole detection.

What additional information do hyperbolic utterances convey beyond their literal counterparts, and how does a listener recover this information? We hypothesize that when people utter a hyperbolic statement, they express an opinion in addition to a description of the state of the world. Hyperbole allows speakers to minimize the cost of an utterance while maximizing the message conveyed. Meanwhile, the listener should be able to infer the additional information embedded in the utterance, namely the true state of the world in addition to the opinion. We will investigate these predictions using a Bayesian computational model and a behavioral experiment. 

%The rest of the paper is organized as follows. Section 2 provides an overview of previous work on hyperbole and pragmatics. Section 3 describes the computational model and its predictions. Section 4 describes the behavioral experiment and results. Section 5 compares the model results to the behavioral data and discusses implications. Section 6 proposes directions for future work.
%%%%

\section{Model}

\subsection{Introduction and Motivation}
We will be building on a traditional approach within linguistics, which views communication as an interaction between rational, cooperative agents \cite{grice1975}. The speaker in a conversation has a meaning to communicate, and her goal is for the listener to understand this meaning. The listener's goal is to infer this intended meaning from the speaker's utterances. The listener performs Bayesian inference to infer the intended meaning, while the speaker is a rational planner who takes into account how the listener will interpret each utterance. 

Recent work \cite{frankgoodmanscience} has proposed a simple formal model of this interaction, in which the listener interprets utterances and the speaker optimizes the informativeness of her utterances for this listener. We will be working with extensions of this model, in which the speaker and listener recursively reason about one another \cite{jager2009pragmatic, bergen2012}. Here, the listener reasons about the speaker optimizing for informativeness; the speaker optimizes given that the listener is reasoning about the speaker; and so on. These models of recursive social reasoning are closely related to the signaling games studied in game theory \cite{cho1987signaling, chen2008selecting}.

\subsection{Pragmatic halo}

We begin by trying to capture the basic ``pragmatic halo" effect: more complex utterances are interpreted as being more precise. Each listener will be associated with a dictionary $\dictionary$, which specifies the literal meaning of each possible utterance. The listener's dictionary determines how he will initially interpret the utterance. All utterances and meanings will be integers in the set $[a,...,b]$. For each utterance $u$, the dictionary entry $\dictionary_u$ for this utterance will be a one-dimensional normal distribution $f(x;u,\sigma^2)$. After hearing the utterance $u$, the listener $L_0$ updates his prior distribution $P$ over meanings \emph{m} by filtering $P$ through the dictionary entry for $u$:
\begin{align}\label{eq:literallistener}
L_0(m | u, \dictionary) &\propto \dictionary_u(m)P(m) \\
&=f(m;u,\sigma^2)P(m).
\end{align}
For modeling pragmatic halo, we can assume that the prior $P$ is uniform over meanings.

The literal listener provides the base case for the recursive social reasoning between the speaker and listener. In general, the speaker $S_n$ is assumed to be a rational planner who is optimizing the probability that her intended meaning \emph{m} will be understood by the listener $L_n$. The listener $L_n$ performs Bayesian inference over the intended meaning given his prior $P$ and his model of the speaker $S_{n-1}$.

The speaker $S_n$ chooses utterances according to a softmax decision rule which describes an approximately rational planner \cite{sutton1998reinforcement}:
\begin{equation}\label{eq:speakerprob}
S_n(u | m,\dictionary) \propto e^{\lambda U_n(u | m,\dictionary)},
\end{equation}
where $\lambda$ is the inverse-temperature. 

The speaker wants to minimize both the cost $c(u)$ of the utterance as well as the information-theoretic surprisal of the intended meaning $m$, so the utility function $U_n$ is defined by:
\begin{equation}\label{eq:speakerutility}
U_n(u | m, \dictionary) = \log (L_{n}(m | u, \dictionary)) - c(u),
\end{equation}
which combined with equation \ref{eq:speakerprob} leads to:
\begin{equation}
S_n(u | m, \dictionary) \propto (L_{n}(m | u,\dictionary)e^{-c(u)}) ^\lambda.
\end{equation}

The speaker $S_0$ reasons about the literal listener $L_0$, and assumes that this listener shares her dictionary $\dictionary$. However, in general the listener will be uncertain about the dictionary being used by the speaker, which we call \emph{lexical uncertainty} \cite{bergen2012}. To determine the speaker's intended meaning, he will therefore marginalize over the possible dictionaries being used:
\begin{equation}
L_n(m|u,\dictionary) \propto \sum_{\dictionary_i }P(m)P(\dictionary_i)S_{n-1}(u | m,\dictionary_i).
\end{equation}
The dictionary $\dictionary$ determines the standard deviation $\sigma_u$ associated with each utterance $u$, therefore specifying how precisely each utterance will be interpreted by the literal listener. Lexical uncertainty represents uncertainty about how precisely the speaker believes her utterances will be interpreted. We assume throughout that the prior probability on dictionaries $P(\dictionary_i)$ is uniformly distributed across the $|S|^|U|$ possible dictionaries, where $S$ is a finite set of possible standard deviations for the utterances and $U$ is the set of possible utterances. 

Because the higher-level listeners marginalize over dictionaries, the dictionary $\dictionary$ plays no role in the reasoning of the listener $L_n(|u,\dictionary)$ or the speaker $S_n(| m, \dictionary)$ for $n>0$, leading us to define:
\begin{equation}
  L_n(m | u) :=  L_n(m | u, \dictionary) \text{ ~~~~~ if $n > 0$}
\end{equation}
\begin{equation}
  S_n(u | m) :=  S_n(u | m, \dictionary) \text{ ~~~~~ if $n > 0$.}
\end{equation}

The model presented here is sufficient to explain the pragmatic halo effect. Consider the simplest possible case, in which the possible meanings are \emph{1} and \emph{2}, and the possible utterances are ``one" and ``two." Suppose that ``two" is much more expensive than ``one." First suppose the speaker wants to communicate \emph{1}. In this case, the speaker will almost never choose to communicate using the utterance ``two." The utterance ``two" is more expensive than the utterance ``one," and its literal meaning is strictly farther away from the speaker's intended meaning. In contrast, suppose the speaker wants to communicate \emph{2}. In this case the two utterances are more evenly balanced, and the speaker may choose the utterance ``one": this utterance is cheaper, though its literal meaning is worse for the speaker than that of ``two." The utterance ``one" will therefore be used by speakers trying to communicate either meaning, while the utterance ``two" will only be used by speakers trying to communicate \emph{2}. It follows that ``two" will be assigned a more precise meaning which is peaked on \emph{2}. 

\subsection{Exaggeration}

We now turn to a different pragmatic effect, \emph{exaggeration}, i.e. the non-literal interpretation of utterances with extreme meanings. Rather than cost, exaggeration is driven by the prior distribution over meanings. Pragmatic halo is the pragmatic effect that results from matched prior probabilities but different utterance costs; exaggeration is the pragmatic effect that results from matched costs but differing prior probabilities. 

The model of exaggeration is nearly identical to the model of pragmatic halo presented in the previous section. The only differences are that we set the cost of the utterances $c(u)=0$, and set the prior distribution over meanings $P$ to be a unimodal distribution on the interval $[a,...,b]$. 

The exaggeration effect follows straightforwardly from this model. Suppose again that there are two meanings, \emph{1} and \emph{2}, and two utterances, ``one" and ``two." Suppose that the meaning \emph{1} is much more likely than \emph{2}. If the speaker believes that the literal meaning of ``two" is vague, then he may use this utterance to communicate the meaning \emph{1} because the listener's priors will bias the interpretation of the utterance towards ``one." In contrast, the speaker would not use the utterance ``one" to communicate the meaning \emph{2}, regardless of how vague its literal meaning is, because the listener's priors will bias the interpretation of the utterance against this meaning. It follows that the utterance ``two" may be interpreted as exaggerated, and as intending to communicate the likely meaning \emph{1}, while the utterance ``one" will be interpreted literally.

\begin{figure}
        \begin{subfigure}[b]{0.5\textwidth}
                \centering
                \includegraphics[width=\textwidth]{model_halo_exaggeration.png}
		\caption{Pragmatic Halo and Exaggeration}
	\end{subfigure}
	 \begin{subfigure}[b]{0.33\textwidth}
                \centering
                %\includegraphics[width=\textwidth]{}
		\caption{can fit more plot here}
	\end{subfigure}
	\caption{The graph on the left shows rah rah rah and the one on the right shows roar roar roar}
\end{figure}

\subsection{Hyperbole}

Hyperbole is similar to exaggeration, except that additional information about the \emph{valence} of the meaning is conveyed. Valence is a second dimension of meaning, separate from the number that the speaker wants to convey. If $V$ is the set of possible valences, then the set of possible meanings $M$ is given by:
\begin{equation}
M = [a,...,b] \times V.
\end{equation}

The model of hyperbole is similar to the exaggeration model, except that it needs to be compatible with meanings that consist of number-valence pairs. The dictionary entry $\dictionary_u$ for an utterance $u$ now consists of a Gaussian centered around $u$, as before, as well as a truth function $T_u:V\rightarrow \{0,1\}$ that determines which valences are compatible with $u$. This leads us to modify equation \ref{eq:literallistener} so that the literal listener is now defined by:

\begin{align}\label{eq:valenceliteral}
L_0((k,v) | u, \dictionary) &\propto \dictionary_u(k,v)P(k,v) \\
&=f(k;u,\sigma^2)T_u(v)P(k)P(v),
\end{align}
where $k$ is the number that the speaker wants to communicate, $v$ is the valence, and we assume for simplicity that number and valence are independent under the prior. 

The rest of the model is extended in a similar manner. We note that there are now many more possible dictionaries: each utterance is assigned both a standard deviation and a truth function, and there are $2^{|V|}$ truth functions on valences. 

We will now illustrate this model using the simplest example of hyperbole. We assume that there are two number meanings, \emph{1} and \emph{2}, and two valences, \emph{neutral} and \emph{negative}, so that there are four pairings of numbers and valences. We assume that \emph{1} is more likely than \emph{2}, and \emph{neutral} is more likely than \emph{negative}. There are two utterances, ``one" and ``two." If the speaker wants to communicate \emph{1}-\emph{neutral}, she is likely to succeed by saying ``one" whether its dictionary entries are vague or precise, because it is the most \emph{a priori} probable meaning. This speaker will assign small probability to the utterance ``two." On the other hand, there are two moderately likely meanings that may lead the speaker to say ``two": \emph{2}-\emph{neutral} and \emph{1}-\emph{negative}. It is clear why the speaker would say ``two" to communicate the first meaning. For the second meaning, \emph{1}-\emph{negative}, the speaker may use the utterance ``two" if she believes that the number meaning of this utterance is vague (and therefore compatible with meaning \emph{1}) and that it uniquely picks out the negative valence. Because ``two" may be used to communicate this meaning, it provides evidence to the listener that this meaning was intended. This is the hyperbolic interpretation of the utterance ``two." 

\subsection{The complete model}

Our final model combines the elements of the previous models. It is intended to simultaneously capture three effects: pragmatic halo, the interpretation of extreme utterances as exaggerated, and the interpretation of exaggerated utterances as hyperbolic. This model will allow the costs of utterances to vary, as in the model of pragmatic halo; allow prior probabilities of meanings to vary, as in the model of exaggeration; and introduce valences into the meaning, as in the model of hyperbole. Formally, the model will be identical to the model of hyperbole, except that we allow for utterance costs $c(u) > 0$. 

\section{Behavioral Experiment}

We conducted a behavioral experiment to test whether humans' interpretation of potentially hyperbolic statements can be explained using the model we proposed. We tested the model on five different scenarios. In each scenario, a speaker makes an utterance that contains a numeric value that conveys information about a particular item or state of the world, for example, the price of a textbook or the temperature outside. Subjects are then asked to interpret the numeric expression.

\subsection{Procedures}

We recruited $220$ subjects located in the United States through Amazon Mechanical Turk. Each subject read five short scenarios in random order regarding the following five domains: the number of minutes a bus is behind schedule, the price of a college textbook, the price of a parking ticket, the number of pages in a reading, and the weather temperature (in degrees Fahrenheit). We selected these domains because we believe people have reliable intuitions about the true distributions of such values, and also because people are likely to exaggerate and express opinions about these issues. $8$ possible values for $X$ were systematically chosen for each scenario. These values consisted of four pairs of two numbers. Each pair contained one round number (e.g. $100$) and a neighboring number that cannot be evenly divided by 10 (e.g. $102$). These neighboring non-round numbers also contain more syllables than their ``rounder" counterparts and are presumably more difficult to utter. The first value pair is close to the mean of the underlying distribution; the second and third value pair are moderately close to the mean; and the fourth value pair is very far from the mean. We chose these values in order to best compare the various effects that we wish to examine as well as how these effects interact. Each scenario was structured in a similar manner, and each subject saw only one of the possible $8$ values for $X$. Table 1 shows an example of the textbook scenario.
\begin{table}[h]
\begin{tabular}{| p{9.7cm}| l |}\hline
\multicolumn{1}{|c|}{\textbf{Scenario}} & \multicolumn{1}{|c|}{\textbf{Values for X}} \\\hline
\multicolumn{1}{|l|}{Ann and Bob are friends. They are taking the same class.} & \multicolumn{1}{|l|}{[$100$, $102$, $150$, $152$,}\\
\multicolumn{1}{|l|}{\textbf{Ann:} ``How much did the textbook cost you?"} & \multicolumn{1}{|l|}{$200$, $202$, $1000$, $1012$]}\\
\multicolumn{1}{|l|}{\textbf{Bob:} ``\{X\} dollars."} & \multicolumn{1}{|l|}{}\\\hline
\multicolumn{1}{|c|}{\textbf{Questions}} & \multicolumn{1}{|c|}{\textbf{Response}} \\\hline
(1) Was Bob being literal about the cost of the textbook, or was he exaggerating? & [Literal / Exaggerating] \\
(2) How much do you think the textbook actually cost? & [Free response] \\
(3) How negative does Bob feel about the cost of the textbook? & [Likert scale]\\\hline
\end{tabular}
\caption{Example scenario on textbook costs}
\label{tab:myfirsttable}
\end{table}

Based on our model and theory of the pragmatic interpretation of numbers, we predict that subjects' responses will exhibit the following phenomena. Firstly, round numbers, or numbers that can be evenly divided by $10$, will be interpreted less precisely and literally. Secondly, less likely numbers, or numbers further away from the mean of the underlying distribution, will be interpreted non-literally�and as closer to the mean. Thirdly, exaggerated utterances will be interpreted as having marked meaning and convey stronger affect. Furthermore, there will be an interaction among these three effects, such that non-round numbers closer to the mean will be interpreted as the most precise and convey the least affect, while round numbers further away from the mean will be interpreted as the most exaggerated and convey the most affect.


\subsection{Results}

% halo and exaggeration

\begin{figure}[t]
        \begin{subfigure}[b]{0.5\textwidth}
                \centering
                \caption{Pragmatic Halo}
		\includegraphics[width=\textwidth]{humans_halo.png}
		
	\end{subfigure}
        \begin{subfigure}[b]{0.5\textwidth}
                \centering
                \caption{Exaggeration}
                \includegraphics[width=\textwidth]{humans_exagg.png}
		
	\end{subfigure}
	\caption{Examples of pragmatic halo and exaggeration across all five domains}
\end{figure}

Of the $220$ subjects we recruited, we excluded $4$ that were non-native English speakers from the analysis. $1$ response from 1 subject contained an obvious typo and was also excluded. 
% Report some stats here?

Results are consistent with our model predictions. As shown in Figure 2, the proportion of subjects who interpreted an utterance as hyperbolic varies as a function of the ``roundness" of the uttered value as well as the distance between the uttered value and the mean of the underlying distribution. In order to examine the pragmatic halo effect, we selected a pair of values for each scenario that is moderately close to the mean of the underlying distribution (Figure 2(a)). We then compared the precision of subjects' interpretation of these numbers, where an interpretation is considered precise if it is identical to the uttered number and imprecise otherwise. For example, if a subject interprets the uttered number ``100" as meaning ``99," then the interpretation is imprecise. Consistent with \cite{someone} findings, we see that number expressions that contain more syllables are more likely to be interpreted precisely, or exactly as they are uttered. In order to examine the effect of prior probability on hyperbolic interpretation, for each scenario we selected a pair of values that contains one non-round number close to the mean of the underlying distribution and one non-round number far from the mean. We then compared how likely subjects are to interpret the numbers as exaggerated, where an exaggerated interpretation is one that is lower than the uttered value. For example, if a subject interprets the uttered number ``1012" as meaning ``100," then ``1012" is interpreted as an exaggeration. 

Subjects are also more likely to judge an utterance as exaggeration when the literal meaning of the numeric value becomes less likely in a particular domain. For instance, when prompting subjects to a temperature of 107 degrees Fahrenheit, they judged it as exaggeration because it is unlikely that the weather gets that high in the United States, from which these subjects were recruited.

% exaggeration and affect (should we still add a plot that just has affect?)
% add something about how affect is computed: median split ( > 5 is affect).

We then analyzed the relationship between exaggeration and speaker's affect. The graphs in Figure 3 show how each utterance is interpreted and the relationship between uttered values and their probability of being perceived to convey affect. In the textbook scenario, the utterance ``100" is most likely to be interpreted as close to the literal meaning and without affective information. On the other hand, the utterance ``1000" is most likely to be interpreted as much lower than the literal meaning and with affective information. We also see a slight effect of pragmatic halo, in which non-round numbers are more likely to be interpreted literally than their nearest round counterparts. 

Interestingly, the proportion of ``affect" responses varies across the five scenarios. In the parking ticket scenario, subjects are asked to judge how negatively a speaker who has just received a parking ticket feels about the ticket cost. Since the speaker is very likely to feel negatively about the ticket cost, the utterances are generally interpreted as having more negative affect than utterances in the textbook scenario. Our model can capture this effect by adjusting the prior probability of having an affective state regarding a particular scenario. 
% plots for all 5 domains
\begin{figure}[t]
        \begin{subfigure}[b]{0.5\textwidth}
                \centering
		\includegraphics[width=\textwidth]{humans_all_textbook.png}
		\caption{textbooks}
	\end{subfigure}
        \begin{subfigure}[b]{0.5\textwidth}
                \centering
                \includegraphics[width=\textwidth]{humans_all_ticket.png}
		\caption{tickets}
	\end{subfigure}
	\begin{subfigure}[b]{0.33\textwidth}
                \centering
                \includegraphics[width=\textwidth]{humans_all_bus.png}
		\caption{bus wait time}
	\end{subfigure}
	\begin{subfigure}[b]{0.33\textwidth}
                \centering
                \includegraphics[width=\textwidth]{humans_all_reading.png}
		\caption{reading assignment}
	\end{subfigure}
	\begin{subfigure}[b]{0.33\textwidth}
                \centering
                \includegraphics[width=\textwidth]{humans_all_weather.png}
		\caption{weather temperature}
	\end{subfigure}
	\caption{Distribution of interpretations for each utterance across five scenarios}
\end{figure}

% everything

\section{Comparison and Discussion}
Our model captures some important intuitions regarding the numerical values and statistical properties of the uttered price. Here we compare our model results and behavioral data on two aspects: (1) Probability of interpreting the utterance as hyperbolic (2) Most likely inferred price given that the utterance is interpreted as hyperbolic or literal.


As shown in Figure 7, the probability that the model interprets the utterance as hyperbolic roughly matches that of the behavioral data, at least in the general trend. 

\begin{figure}
        \begin{subfigure}[b]{0.5\textwidth}
                \centering
		\includegraphics[width=\textwidth]{model_all_textbook.png}
		\caption{model all textbook}
	\end{subfigure}
        \begin{subfigure}[b]{0.5\textwidth}
                \centering
                \includegraphics[width=\textwidth]{humans_all_textbook.png}
		\caption{humans all textbook}
	\end{subfigure}
	\caption{The graph on the left shows rah rah rah and the one on the right shows roar roar roar}
\end{figure}

\section{Future Directions \& Conclusion}

%\subsubsection*{References}
%
%\small{
%[1] Alexander, J.A. \& Mozer, M.C. (1995) Template-based algorithms
%for connectionist rule extraction. In G. Tesauro, D. S. Touretzky
%and T.K. Leen (eds.), {\it Advances in Neural Information Processing
%Systems 7}, pp. 609-616. Cambridge, MA: MIT Press.
%
%[2] Bower, J.M. \& Beeman, D. (1995) {\it The Book of GENESIS: Exploring
%Realistic Neural Models with the GEneral NEural SImulation System.}
%New York: TELOS/Springer-Verlag.
%
%[3] Hasselmo, M.E., Schnell, E. \& Barkai, E. (1995) Dynamics of learning
%and recall at excitatory recurrent synapses and cholinergic modulation
%in rat hippocampal region CA3. {\it Journal of Neuroscience}
%{\bf 15}(7):5249-5262.
%}

\bibliographystyle{apacite}

\setlength{\bibleftmargin}{.125in}
\setlength{\bibindent}{-\bibleftmargin}

\bibliography{nips_hyperbole}


\end{document}
